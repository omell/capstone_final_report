\section{Implementation - firmware and software}
%%%%%%%%%%%%%%%%%%%%%%%%%%%%%%%%%%%%%%%%%%%%%%%%%%%%%%%%%%%%%%%%%%%%%%%%%%%%%%%%
\begin{comment}
    \hl{talk about the limitation on speed  - discussed in discussion section}
    modules required on microcontroller for product to satisfy objectives of Section~\ref{sec:objectives}
    \begin{itemize}
        \item timer module for guaranteed execution of control loop regime at \hl{desired} frequency
        \item module to handle reading and processing of DC-DC converter circuit voltages
        \item timer module for microcontroller to transmit to user interface
        \item communication module for microcontroller to receive from user interface
    \end{itemize}
    \hl{in implementation section explain timer modules in detail (frequency, configuration etc.) (timer.c); explain UART.c; explain ADC.c; explain PWM.c; explain main.c}
\end{comment}
%%%%%%%%%%%%%%%%%%%%%%%%%%%%%%%%%%%%%%%%%%%%%%%%%%%%%%%%%%%%%%%%%%%%%%%%%%%%%%%%

The MCU has two main responsibilities; the computation of the control signal and interfacing with the users as discussed in \hl{Section XXX}. This section outlines the implementation of controller and user interface on the MCU. 


\subsection{Controller Architecture}
The controller implemented includes integral action and the Luenberger observer. However only the integral action was used to compute the control signal. The flowchart of the control loop is provided in Figure~\ref{flow:timerISR}.

%%%%%%%%%%%%%%%%%%%%%%%%%%%%%%%%%%%%%%%%%%%%%%%%%%%%%%%%%%%%%%%%%%%%%%%%%%%%%%%%
\tikzstyle{decision} = [diamond, draw, fill=\myblue, 
    text width=4.5em, text badly centered, node distance=3cm, inner sep=0pt]
\tikzstyle{block} = [rectangle, draw, fill=\myblue, 
    text width=15em, text badly centered, rounded corners, minimum height=4em]
\begin{figure}[H]
    \centering
    \fbox{
    \begin{tikzpicture}[node distance = 2.5cm]
    \node [block] (1) {Timer interrupt occurs};
    \node [block, below of = 1] (1b) {Update the duty cycle that is being sent to the converter};
    \node [block, below of = 1b] (3) {Take measurements of voltage across ultra-capacitors and load voltages};
    \node [block, below of = 3] (3b) {Calculate new duty cycle based on measurements and estimates};
    \node [block, below of = 3b] (3c) {Make new estimates of other circuit variables};
    \node [block, below of = 3c] (4) {Return to main and wait};
    % label
    \node [rectangle, draw, fill = white, text width=15em, text centered, minimum height = 4em, above of = 1] (-1) {interrupt-based control regime\\operates at $4 \ \mathsf{kHz}$ \\(triggers every $250 \ \mathsf{\mu s}$)};
    % edges
    \path [line, ultra thick] (1) -- (1b);
    \path [line, ultra thick] (1b) -- (3);
    \path [line, ultra thick] (3) -- (3b);
    \path [line, ultra thick] (3b) -- (3c);
    \path [line, ultra thick] (3c) -- (4);
    \path [line, ultra thick] (4) -- ($(4) + (4,0)$) -- ($(1) + (4,0)$) -- (1);
    \end{tikzpicture}
    }
    \caption{Controller flowchart}
    \label{flow:timerISR}
\end{figure}
%%%%%%%%%%%%%%%%%%%%%%%%%%%%%%%%%%%%%%%%%%%%%%%%%%%%%%%%%%%%%%%%%%%%%%%%%%%%%%%%

Timer was used to run a control loop at exactly 4kHz as it guarantees that this loop gets executed at the exact frequency every time. \\

When the timer interrupt occurs, it triggers ADC to take samples at the ultra-capacitors and at the output simultaneously, as well as it calls timer Interrupt Service Routine (ISR). In ISR, it updates the duty cycle, which is calculated from the previous sampling interval. ADC takes some time to convert the data sampled, once the conversion is complete, a new duty cycle is computed using integral action regime. The estimates of circuit variables are then calculated using state feedback regime. When all the computation is complete, it returns to main. \\

This order is crucial, especially the duty cycle being updated at the beginning of the control loop. This is because when the controller was discretized it was assumed that all of the control variables are constant over a sampling period \hl{refer to section}. Hence the duty cycle must not be updated middle of the control loop nor when the computation is complete.\\

The execution time of this loop was tested by toggling LEDs, and it was found to be roughly 4.2kHz. The control loop was set to 4kHz, which allows UART communication after computation is complete. The execution time will be further discussed in \hl{Section XXX}. 
\subsubsection{Configuring System clock}
The MCU has internal system clock running at 7.3728MHz and the frequency of the instruction clock is the half of the system clock. Since execution time is critical, built-in Phase Lock Loop (PLL) was used to increase the frequency of the system clock. The system clock was configured to frequency of 120MHz and the instruction frequency of 60MHz. 
%%%%%%%%%%%%%%%%%%%%%%%%%%%%%%%%%%%%%%%%%%%%%%%%%%%%%%%%%%%%%%%%%%%%%%%%%%%%%%%%
\begin{comment}
    Timer 2 was configured as ADC triggering is available on TMR2. Timer count register gets incremented every rising edge of timer input clock. It is configured to take the internal clock running at 60MHz as timer source. The period of timer interrupt is controlled by PR2 (16bit) register. In order to trigger the interrupt at 10kHz, PR2 is set to:
    \begin{align*}
        PR2 
        & = \frac{F_{CY}}{PCLKDIV \times f_{desired}} \\
        & =  \frac{60MHz}{1 \times 10kHz} \\
        & = 6000 \text{ instruction cycles}
    \end{align*}
    where:
    $F_{CY}$ is the instruction clock cycle, which is half of system clock i.e. 60MHz\\
    $PCLKDIV$ is the input pre-scalar and is set to 1 providing 1:1 ratio.  
\end{comment}    
%%%%%%%%%%%%%%%%%%%%%%%%%%%%%%%%%%%%%%%%%%%%%%%%%%%%%%%%%%%%%%%%%%%%%%%%%%%%%%%%

\subsubsection{Computation of control signal}
The control signal is computed based on integral action and forward Euler method was implemented as discussed in Section \ref{sec:disc}. It takes the voltage at load from the previous and current time-step to update the control signal. The transfer function is given by:
\[
    H(z) = \frac{T_s}{z-1}
\]
The difference equation of this transfer function is given by:
\[
    y[k] - y[k-1] = T_s x[k] \: \Longrightarrow \: y[k] = y[k-1] + T_s x[k]
\]
where $y[k]$ is the control signal and $x[k]$ is the difference between the current output voltage and reference, $k$ is the time step and $T_s$ is the sampling interval (250$\mu$s).

This control signal is then get multiplied by the constant gain determined by LRQ, discussed in Section XXX. 
\hl{Maybe worth talking about the performance changing gain in result?}
\newpar
When there is a large change in a reference point, the integrator gets saturated. Once it reaches saturation, the integral term accumulates and keep increasing. This would cause large delay in the system. Hence anti-windup was implemented to avoid actuator saturation. After a new control signal is calculated, the magnitude of the control signal was checked. When the signal is too large or too small, it gets clipped to the maximum, 80\% or the minimum of 0\% and the excess is stored as \lq anti-windup\lq. When there is non-zero \lq anti-windup\lq from previous sampling period, this is subtracted from the control signal. 

\subsubsection{PWM as control signal}
The controller calculates the duty cycle of PWM signal to be sent to the converter. The frequency of the duty cycle is set to 100kHz. This is because higher switching frequency permits lower inductance L1 according to continuous conduction mode condition \hl{SECTION?}, which allows smaller and cheaper components. However higher frequency increases loss in resistive elements. 

\begin{comment}
  higher switching frequency permits lower inductance L1 (according to continuous conduction mode condition; i.e. allows smaller and cheaper components) but increases loss (in resistive elements). also higher switching frequency increases accuracy of state-space averaging technique (\cite{cuk})(may write something up on that; has to do with how L2 and C2 are essentially a lowpass filter and how the accuracy of state-space averaging depends on how much smaller the frequency of the filter they describe is than the the switching frequency)
\end{comment}

The duty cycle of PWM is controlled by PDC1 register. It is determined by the formula\cite{picPWM}:
\[
    PDC1 = \left( \frac{ACLK\times 8\times \text{Desired duty cycle in second}}{Prescalar (PCLKDIV)} \right)
\]
where:\\
$ACLKL$ is the frequency of Auxiliary clock, which is set to 60MHz\\
$PCLKDIV$ is input clock pre-scalar and is set to 1 providing 1:1 ratio. \\

To reduce the number of computations, the constant term was definite. The dudy cycle is  updated with the formula:
\[
    PDC1 = \text{duty cycle}\cdot \underbrace{(60\times10^6\times 8\times \text{PWM period})}_{\text{defined as constant}}
\]


\subsection{Interface}

\subsubsection{Transmitting}

\subsubsection{Receiveing}
When information is received at MCU from a host machine, MCU was configured to set an interrupt flag and received byte is handled in UART Receive ISR. This interrupt was prioritized lower than the control loop so when a byte is received it does not interrupt the control loop and the receive interrupt is executed after control loop is complete.

On the MCU, there are 4 deep First In First Out(FIFO) deceived data buffers. When the information is received, all the bytes are stored in the buffer and the interrupt handles one byte at a time.


%%%%%%%%%%%%%%%%%%%%%%%%%%%%%%%%%%%%%%%%%%%%%%%%%%%%%%%%%%%%%%%%%%%%%%%%%%%%%%%%
\subsection{Interrupt priorities}
For the microcontroller to be able to handle different task without interrupting the others, priority was assigned to each, which is illustrated in the Figure \ref{fig:priority}.

\tikzstyle{decision} = [diamond, draw, fill=\myblue, 
    text width=4.5em, text badly centered, node distance=3cm, inner sep=0pt]
\tikzstyle{block} = [rectangle, draw, fill=\myblue, 
    text width=9em, text badly centered, rounded corners, minimum height=6em]
\tikzstyle{line} = [draw, -latex']
\tikzstyle{cloud} = [draw, ellipse,fill=red!20, node distance=3cm,
    minimum height=2em]
\setlength{\fboxsep}{10pt}
\begin{figure}[H]
\fbox{
\centering
\begin{tikzpicture}[node distance = 1cm, auto]
\node [block, text width = 8em] (1) {timer2 ($10 \ \mathsf{kHz}$) \\ control loop; \\ ADCs};
\node [block, right = of 1, text width = 8em] (2) {timer1 ($10 \ \mathsf{Hz}$) \\ MCU transmit \\ $y$, $v_g$, $d$};
\node [block, right = of 2, text width = 8em] (3) {UARTRX \\ MCU receive \\ \textsf{ref}};
% timer ISR label
\node [rectangle, draw, fill = white, text width=12em, text centered, minimum height = 5em, above = of 2] (-1) {ISR priority configuration};
% lines
\draw[color = white] (1) -- node [anchor = center] {\color{black}\Huge$>$} (2);
\draw[color = white] (2) -- node [anchor = center] {\color{black}\Huge$>$} (3);
% border
%\node [block, text width=5em, minimum height=6em] (1) {$<$};
\end{tikzpicture}
}
\caption{}
\label{fig:priority}
\end{figure}
%%%%%%%%%%%%%%%%%%%%%%%%%%%%%%%%%%%%%%%%%%%%%%%%%%%%%%%%%%%%%%%%%%%%%%%%%%%%%%%%