\section{implementation/experimentation/simulation/testing/results}
%%%%%%%%%%%%%%%%%%%%%%%%%%%%%%%%%%%%%%%%%%%%%%%%%%%%%%%%%%%%%%%%%%%%%%%%%%%%%%%%
%%%%%%%%%%%%%%%%%%%%%%%%%%%%%%%%%%%%%%%%%%%%%%%%%%%%%%%%%%%%%%%%%%%%%%%%%%%%%%%%
%%%%%%%%%%%%%%%%%%%%%%%%%%%%%%%%%%%%%%%%%%%%%%%%%%%%%%%%%%%%%%%%%%%%%%%%%%%%%%%%
\subsection{Implementation}
%%%%%%%%%%%%%%%%%%%%%%%%%%%%%%%%%%%%%%%%%%%%%%%%%%%%%%%%%%%%%%%%%%%%%%%%%%%%%%%%
%%%%%%%%%%%%%%%%%%%%%%%%%%%%%%%%%%%%%%%%%%%%%%%%%%%%%%%%%%%%%%%%%%%%%%%%%%%%%%%%
%%%%%%%%%%%%%%%%%%%%%%%%%%%%%%%%%%%%%%%%%%%%%%%%%%%%%%%%%%%%%%%%%%%%%%%%%%%%%%%%
\subsection{Experimentation}
%%%%%%%%%%%%%%%%%%%%%%%%%%%%%%%%%%%%%%%%%%%%%%%%%%%%%%%%%%%%%%%%%%%%%%%%%%%%%%%%
%%%%%%%%%%%%%%%%%%%%%%%%%%%%%%%%%%%%%%%%%%%%%%%%%%%%%%%%%%%%%%%%%%%%%%%%%%%%%%%%
%%%%%%%%%%%%%%%%%%%%%%%%%%%%%%%%%%%%%%%%%%%%%%%%%%%%%%%%%%%%%%%%%%%%%%%%%%%%%%%%
\subsection{Simulation}
Simulations performed on system specified by parameters in Tables~\ref{tab:cuk_values} and~\ref{tab:system_values}.
%%%%%%%%%%%%%%%%%%%%%%%%%%%%%%%%%%%%%%%%%%%%%%%%%%%%%%%%%%%%%%%%%%%%%%%%%%%%%%%%
%%%%%%%%%%%%%%%%%%%%%%%%%%%%%%%%%%%%%%%%%%%%%%%%%%%%%%%%%%%%%%%%%%%%%%%%%%%%%%%%
\subsubsection{Specifications}
State-space description of system (continuous time)
\begin{align*}
\boldsymbol{A}
&=
\begin{bmatrix}
\texttt{0} & \texttt{0} & \texttt{+4.166667E+04} & \texttt{-4.166667E+04}\\
\texttt{0} & \texttt{-1.218537E+03} & \texttt{0} & \texttt{+1.218537E+05}\\
\texttt{-1.063830E+03} & \texttt{0} & \texttt{-1.816915E+03} & \texttt{-1.421277E+03}\\
\texttt{-1.063830E+03} & \texttt{-2.125959E+03} & \texttt{-1.421277E+03} & \texttt{-1.195715E+03}
\end{bmatrix}
\\[11pt]
\boldsymbol{b}
&=
\begin{bmatrix}
\texttt{0} & \texttt{0} & \texttt{0} & \texttt{0}\\
\texttt{0} & \texttt{0} & \texttt{0} & \texttt{0}\\
\texttt{+2.127660E+03} & \texttt{-1.063830E+03} & \texttt{0} & \texttt{0}\\
\texttt{0} & \texttt{-1.063830E+03} & \texttt{0} & \texttt{0}
\end{bmatrix}
\\[11pt]
\boldsymbol{c}
&=
\begin{bmatrix}
\texttt{0} & \texttt{+9.992006E-01} & \texttt{0} & \texttt{+7.993605E-02}
\end{bmatrix}
\\[11pt]
\boldsymbol{b}_d
&=
\begin{bmatrix}
\texttt{+8.364441E+03}\\
\texttt{0}\\
\texttt{+2.132688E+04}\\
\texttt{-2.034445E+04}
\end{bmatrix}
\end{align*}
%%%%%%%%%%%%%%%%%%%%%%%%%%%%%%%%%%%%%%%%%%%%%%%%%%%%%%%%%%%%%%%%%%%%%%%%%%%%%%%%
Note that this system is controllable from $\hat{d}$, which may be checked using the \textsf{MATLAB} call of~(\ref{matlab:controllable}) for our 4 state system. There are thus no uncontrollable states and thus $(\boldsymbol{A}, \, \boldsymbol{B})$ is stabilizable from $\hat{d}$ (a condition required to employ LQR control design).
\begin{align}\label{matlab:controllable}
\texttt{rank(ctrb(A, bD)) == 4}
\end{align}
% ~\rule{\textwidth}{0.5pt}
%%%%%%%%%%%%%%%%%%%%%%%%%%%%%%%%%%%%%%%%%%%%%%%%%%%%%%%%%%%%%%%%%%%%%%%%%%%%%%%%
State-space description of system (discrete time)
\begin{align*}
\overline{\squarey{\boldsymbol{A} - \boldsymbol{L}\boldsymbol{c}}}
&=
\begin{bmatrix}
\texttt{+5.899444E-04} & \texttt{-6.010730E-02} & \texttt{+5.768492E-04} & \texttt{+6.223055E-02}\\
\texttt{+1.155143E-06} & \texttt{-1.070972E-04} & \texttt{+1.115969E-06} & \texttt{+1.171403E-04}\\
\texttt{-1.052231E-03} & \texttt{+1.040045E-01} & \texttt{-1.024871E-03} & \texttt{-1.095839E-01}\\
\texttt{+5.326313E-06} & \texttt{-5.042553E-04} & \texttt{+5.159615E-06} & \texttt{+5.448561E-04}
\end{bmatrix}
\\[11pt]
\overline{\boldsymbol{b}}
&=
\begin{bmatrix}
\texttt{+3.806479E-02} & \texttt{-9.994101E-01} & \texttt{0} & \texttt{0}\\
\texttt{-1.942447E-05} & \texttt{+1.155143E-06} & \texttt{0} & \texttt{0}\\
\texttt{-9.973269E-03} & \texttt{-1.052231E-03} & \texttt{0} & \texttt{0}\\
\texttt{-1.151095E-04} & \texttt{+5.326313E-06} & \texttt{0} & \texttt{0}
\end{bmatrix}
\\[11pt]
\overline{\boldsymbol{c}}
&=
\begin{bmatrix}
\texttt{0} & \texttt{+9.992006E-01} & \texttt{0} & \texttt{+7.993605E-02}
\end{bmatrix}
\\[11pt]
\overline{\boldsymbol{b}_d}
&=
\begin{bmatrix}
\texttt{-1.809741E+01}\\
\texttt{-3.591587E-04}\\
\texttt{-4.164968E-01}\\
\texttt{-2.157311E-03}
\end{bmatrix}
\end{align*}
~\rule{\textwidth}{0.5pt}
LQR
\begin{itemize}
    \item \texttt{w = 1}
    \item \texttt{q = diag([0 0 0 0 100])}
    \item \texttt{r = 1}
\end{itemize}
returns
\begin{itemize}
    \item \texttt{Kf = [+9.671365E-05 +5.727858E-05 +4.957778E-03 -4.808127E-03]}
    \item \texttt{Ki = -10.000000}
\end{itemize}
~\rule{\textwidth}{0.5pt}
Observer
\begin{itemize}
    \item \texttt{p = -1.416057E+05}
    \item \texttt{L = [+9.671365E-05 +5.727858E-05 +4.957778E-03 -4.808127E-03]}
    \item \texttt{L\_bar = [-1.924405E+00, +1.000089E+00, -1.049633E-01, +1.039485E-02]}
\end{itemize}
Note that this observer system produces stable $\overline{\squarey{\boldsymbol{A} - \boldsymbol{L}\boldsymbol{c}}}$ (a requirement for the observer to generate meaningful estimates of state variables).
%%%%%%%%%%%%%%%%%%%%%%%%%%%%%%%%%%%%%%%%%%%%%%%%%%%%%%%%%%%%%%%%%%%%%%%%%%%%%%%%
%%%%%%%%%%%%%%%%%%%%%%%%%%%%%%%%%%%%%%%%%%%%%%%%%%%%%%%%%%%%%%%%%%%%%%%%%%%%%%%%
% \subsubsection{Method}
% Notes:
% \begin{itemize}
%     \item ADC conversions and ZOH-discretisation simulated through \textsf{Simulink}'s ZOH blocks
% \end{itemize}
%%%%%%%%%%%%%%%%%%%%%%%%%%%%%%%%%%%%%%%%%%%%%%%%%%%%%%%%%%%%%%%%%%%%%%%%%%%%%%%%
%%%%%%%%%%%%%%%%%%%%%%%%%%%%%%%%%%%%%%%%%%%%%%%%%%%%%%%%%%%%%%%%%%%%%%%%%%%%%%%%
\subsubsection{\textsf{Simulink}}
\begin{figure}[H]
    \centering
    \fbox{
    \includegraphics[width = 0.95\textwidth]{figures/simulink.pdf}
    }
    \caption{\textsf{Simulink} model used to simulate the \'{C}uk converter under ultimate control regime of Section~\ref{sec:control_ultimate} (simplified)}
    \label{fig:simulink}
\end{figure}
Notes:
\begin{itemize}
    \item PWM generator set to switching frequency $f_\text{switching} = 100 \ \mathsf{kHz}$
    \item ZOH blocks configured for sample time \texttt{Ts = 100e-6}
    \item \texttt{ALc\_bar} denotes the matrix $\overline{\squarey{\boldsymbol{A} - \boldsymbol{L}\boldsymbol{c}}}$
\end{itemize}
%%%%%%%%%%%%%%%%%%%%%%%%%%%%%%%%%%%%%%%%%%%%%%%%%%%%%%%%%%%%%%%%%%%%%%%%%%%%%%%%
\begin{figure}[H]
    \centering
    \fbox{
    \includegraphics[height = 0.4\textheight]
    {figures/estimation/ref_y_pulsetrain.pdf}
    }
    \caption{Simulated system response to step changes in reference}
    \label{fig:simulation_pulsetrain}
\end{figure}
Some quantities of note from this simulation are provided in Table~\ref{tab:simulation_step}. Quantities maximum unless specified otherwise.
\begin{table}[H]
    \centering
    \begin{tabular}{|c|c|c|c|}
    \hline
    Quantity & Value [S.I.] & Value [\%] & Notes\\
    \hline
    Overshoot & $190 \ \mathsf{mV}$ & $1.2$ & at $\textsf{ref: } \minus 10 \rightarrow \minus 16 \ \mathsf{V}$\\
    \hline
    Output voltage ripple (boost mode) & $21 \ \mathsf{mV}$ & $0.13$ & at $\textsf{ref } = \minus 16 \ \mathsf{V}$\\
    \hline
    Output voltage ripple (buck mode) & $5.4 \ \mathsf{mV}$ & $0.14$ & at $\textsf{ref } = \minus 1 \ \mathsf{V}$\\
    \hline
    Steady-state error & $0$ & $0$ & for these steps $100 \ \mathsf{ms}$ apart\\
    \hline
    \end{tabular}
    \caption{}
    \label{tab:simulation_step}
\end{table}
Additional notes:
\begin{itemize}
    \item transient behaviour upon system turn on discounted in populating Table~\ref{tab:simulation_step}
    \item settling time maximum for reference undergoing step change from greater to lesser magnitude
    \item output takes $\approx 100 \ \mathsf{ms}$ to settle for $\textsf{ref: } \minus 16 \rightarrow \minus 1 \ \mathsf{V}$
\end{itemize}
%%%%%%%%%%%%%%%%%%%%%%%%%%%%%%%%%%%%%%%%%%%%%%%%%%%%%%%%%%%%%%%%%%%%%%%%%%%%%%%%
\begin{figure}[H]
    \centering
    \fbox{
    \includegraphics[height = 0.4\textheight]
    {figures/estimation/ref_y_ramp.pdf}
    }
    \caption{Simulated system response to ramp disturbance in input voltage at fixed reference}
    \label{fig:simulation_ramp}
\end{figure}
Some quantities of note from this simulation are provided in Table~\ref{tab:simulation_ramp}. Quantities maximum unless specified otherwise.
\begin{table}[H]
    \centering
    \begin{tabular}{|c|c|c|c|}
    \hline
    Quantity & Value [S.I.] & Value [\%] & Notes\\
    \hline
    Output voltage ripple & $71 \ \mathsf{mV}$ & $1.4$ & Decreases as $v_g$ decreases\\
    \hline
    Steady-state error & $9 \ \mathsf{mV}$ & $0.18$ & Increases rapidly as $v_g$ drops below $0.8 \ \mathsf{V}$\\
    \hline
    \end{tabular}
    \caption{}
    \label{tab:simulation_ramp}
\end{table}
Additional notes:
\begin{itemize}
    \item transient behaviour upon system turn on discounted in populating Table~\ref{tab:simulation_ramp}
    \item $\textsf{ref } = \minus 5 \ \mathsf{V}$ $\implies$ duty ratio saturates at $90 \ \mathsf{\%}$ for $v_g < 0.8 \ \mathsf{V}$
\end{itemize}
%%%%%%%%%%%%%%%%%%%%%%%%%%%%%%%%%%%%%%%%%%%%%%%%%%%%%%%%%%%%%%%%%%%%%%%%%%%%%%%%
%%%%%%%%%%%%%%%%%%%%%%%%%%%%%%%%%%%%%%%%%%%%%%%%%%%%%%%%%%%%%%%%%%%%%%%%%%%%%%%%
%%%%%%%%%%%%%%%%%%%%%%%%%%%%%%%%%%%%%%%%%%%%%%%%%%%%%%%%%%%%%%%%%%%%%%%%%%%%%%%%
\subsection{Testing}
Testing revealed that the gate driver IC did not function as anticipated. It was decided to bypass this IC and drive the \'{C}uk converter MOSFET directly from the microcontroller. This was the configuration used in testing performance of converter circuit powered from microcontroller PWM module.
%%%%%%%%%%%%%%%%%%%%%%%%%%%%%%%%%%%%%%%%%%%%%%%%%%%%%%%%%%%%%%%%%%%%%%%%%%%%%%%%
\subsubsection{Open-loop - characterising transient response}
Testing on \'{C}uk converter begun was observing the transient response at three operation points with $100 \ \mathsf{\Omega}$ resistor as load. The results are provided in Table~\ref{tab:trans}. Simulated output extracted from \textsf{MATLAB} script of Appendix~\ref{apx:MATLAB}.
\begin{table}[H]
\centering
\begin{tabular}{|c|c|c||c|}
\hline
Duty cycle  & Output voltage & Ripple  & Simulated output \\ \hline \hline
12.5\%      & -925mV         & 52mV     & -975mV \\ \hline
50\%        & -4.5V          & 68mV     & -5.02V\\\hline
62.5 \%     & -12.8V         & 63mV     & -8.38V\\ \hline
\end{tabular}
\caption{Transient response}
\label{tab:trans}
\end{table}
Worst case settling time was found to be $\approx 10 \ \mathsf{ms}$ and was produced for $y \text{: } \minus 0.925 \rightarrow \minus 12.8 \ \mathsf{V}$.

%%%%%%%%%%%%%%%%%%%%%%%%%%%%%%%%%%%%%%%%%%%%%%%%%%%%%%%%%%%%%%%%%%%%%%%%%%%%%%%%
\subsubsection{Open-loop - mapping duty ratio to output voltage}
\begin{figure}[H]
\centering
\fbox{\includegraphics[height = 0.4\textheight]{converter_openloop.pdf}}
\caption{}
\label{fig:steps}
\end{figure}
%%%%%%%%%%%%%%%%%%%%%%%%%%%%%%%%%%%%%%%%%%%%%%%%%%%%%%%%%%%%%%%%%%%%%%%%%%%%%%%%
\subsubsection{Testing operating region limits}
A PWM signal of duty ratio $85 \ \mathsf{\%}$ was sent to the converter to test stability, safety, and power usage. A summary of the data produced in this experiment is provided in Table~\ref{tab:heating}.
\begin{table}[H]
    \centering
    \begin{tabular}{|c|c|c|}
    \hline
    Quantity & Value [S.I.] & Notes\\
    \hline
    $d$ & $85 \ \mathsf{\%}$ &\\
    \hline
    $v_g$ & $5 \ \mathsf{V}$ &\\
    \hline
    Output voltage ripple & $75 \ \mathsf{mV} \ (0.4 \ \mathsf{\%})$ &\\
    \hline
    Input current draw & $1.15 \ \mathsf{A}$&\\
    \hline
    \end{tabular}
    \caption{}
    \label{tab:heating}
\end{table}
Under these conditions the $100 \ \mathsf{\Omega}$ resistor dissipated $4 \ \mathsf{W}$ of heat (and became too hot to touch).
%%%%%%%%%%%%%%%%%%%%%%%%%%%%%%%%%%%%%%%%%%%%%%%%%%%%%%%%%%%%%%%%%%%%%%%%%%%%%%%%
%%%%%%%%%%%%%%%%%%%%%%%%%%%%%%%%%%%%%%%%%%%%%%%%%%%%%%%%%%%%%%%%%%%%%%%%%%%%%%%%
%%%%%%%%%%%%%%%%%%%%%%%%%%%%%%%%%%%%%%%%%%%%%%%%%%%%%%%%%%%%%%%%%%%%%%%%%%%%%%%%
\subsection{Controller}
Collection of data to present performance of control regime in hardware is ongoing. Results are not in a state to present in this report at this time.
%%%%%%%%%%%%%%%%%%%%%%%%%%%%%%%%%%%%%%%%%%%%%%%%%%%%%%%%%%%%%%%%%%%%%%%%%%%%%%%%
%%%%%%%%%%%%%%%%%%%%%%%%%%%%%%%%%%%%%%%%%%%%%%%%%%%%%%%%%%%%%%%%%%%%%%%%%%%%%%%%
%%%%%%%%%%%%%%%%%%%%%%%%%%%%%%%%%%%%%%%%%%%%%%%%%%%%%%%%%%%%%%%%%%%%%%%%%%%%%%%%