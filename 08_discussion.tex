\section{Discussion}
%%%%%%%%%%%%%%%%%%%%%%%%%%%%%%%%%%%%%%%%%%%%%%%%%%%%%%%%%%%%%%%%%%%%%%%%%%%%%%%%
%%%%%%%%%%%%%%%%%%%%%%%%%%%%%%%%%%%%%%%%%%%%%%%%%%%%%%%%%%%%%%%%%%%%%%%%%%%%%%%%
\subsection{Physical Modelling}
\subsubsection{Effect of a change in load resistance \hl{TODO finish}}\label{sec:varyload}
A typical requirement of DC-DC converters is for operating over a wide range of load resistances. The load resistance required for the \'Cuk converter to perform \hl{smoothly/effectively} is defined in lower bound by \hl{stability}: load resistance $R$ appears in the \hl{denominator} of Equations~(\ref{eqn:A_avg}) through~(\ref{eqn:c_avg}).
\newpar
The upper bound on load resistance is determined by the continuous conduction mode condition:
\begin{align}
\frac{(1 - d)^2 R T}{2 d} < L_1. \label{eqn:conductionmode}
\end{align}
For the operating parameters/component values of Tables~\ref{tab:cuk_values} through~\ref{tab:system_values}, the upper bound on load resistance $R$ is
\begin{align*}
R < \frac{2 d L_1}{(1 - d)^2 T} = \frac{2 \cdot 0.8 \cdot 470 \times 10^{-6}}{(1 - 0.8)^2 \cdot 10 \times 10^{-6}}.
\end{align*}
\hl{TODO explain function of d (where maximum etc.)}
%%%%%%%%%%%%%%%%%%%%%%%%%%%%%%%%%%%%%%%%%%%%%%%%%%%%%%%%%%%%%%%%%%%%%%%%%%%%%%%%
%%%%%%%%%%%%%%%%%%%%%%%%%%%%%%%%%%%%%%%%%%%%%%%%%%%%%%%%%%%%%%%%%%%%%%%%%%%%%%%%
\subsection{Control Design}
% \hl{TODO: recommendations for future projects}
\subsubsection{Generating estimates of state variables}\label{sec:observing}
The observability matrix of our system has full rank, implying that there are no unobservable states. This fact by itself is limited in its insight however: another element affecting the accuracy of an observer regime is the condition number of the observability matrix.
\newpar
With the component values specified in Table~\ref{tab:cuk_values}:
\begin{align}
\texttt{cond(obsv(A, c)) = 5.5831e+13}
\end{align}
which is much larger than 1, indicating an ill-conditioned observability matrix. This suggests poor accuracy of the estimates produced by our observer regime.
\newpar
In practice, it was found that good (to be quantified in following paragraphs) estimates of states $v_{C2}$ and $i_{L2}$ could be extracted with an observer regime. States $v_{C1}$ and $i_{L1}$ were not able to be estimated to reasonable precision. Simulation data in support of this behaviour is provided in Figures~\ref{fig:estimating} and~\ref{fig:estimating2}.
\newpar
$v_{C2}$: output voltage $v_o$ is almost exactly $v_{C2}$ ($v_o$ slightly larger due to voltage drop across $C_2$ parasitic resistance $R_4$). No difference between estimated $v_{C2}$ in \textsf{Simulink} or hardware (i.e. reading estimates microcontroller returns over UART) and simulated $v_o$ could be observed to 3 decimal places for $\textsf{ref: } \minus 16 \ \mathsf{V}$.
% Note that in below equations $\sim$ denotes that the quantity \hl{ripples} between the upper and lower values.
\newpar
\hl{TODO explain tables; explanation of estimates for $i_{L2}$}
\begin{table}[H]
    \centering
    \begin{tabular}{|c|c|c|c|}
    \hline
    $\textsf{ref} \ [\mathsf{V}]$ & $i_{L2}$ (simulated, steady-state value) $[\mathsf{A}]$ & $i_{L2}$ (estimated, \textsf{Simulink}) $[\mathsf{A}]$ & Error $[\mathsf{\%}]$\\
    \hline
    $\minus 16$ & $\minus 0.1616$ & $\minus 0.15948$ & $1.3$\\
    \hline
    \end{tabular}
    \caption{}
    \label{tab:estimating_iL2_ref16}
\end{table}
\begin{table}[H]
    \centering
    \begin{tabular}{|c|c|c|c|}
    \hline
    $\textsf{ref} \ [\mathsf{V}]$ & $i_{L2}$ (simulated, steady-state value) $[\mathsf{A}]$ & $i_{L2}$ (estimated, hardware) $[\mathsf{A}]$ & Error $[\mathsf{\%}]$\\
    \hline
    $\minus 2$ & $\minus 19.89 \times 10^{-3}$ & $\minus 19.9 \times 10^{-3}$ & $0.07$\\
    \hline
    \end{tabular}
    \caption{Estimated $i_{L2}$ in this test was extracted from observer running on microcontroller by printing the value over UART to a connected laptop}
    \label{tab:estimating_iL2_ref2}
\end{table}
%%%%%%%%%%%%%%%%%%%%%%%%%%%%%%%%%%%%%%%%%%%%%%%%%%%%%%%%%%%%%%%%%%%%%%%%%%%%%%%%
\begin{comment}
\begin{align*}
i_{L2}: \textsf{ref: } \minus 16 \ \mathsf{V} \implies &i_{L2}(\text{simulated}): \minus 0.1211 \sim \minus 0.20205 \ \mathsf{A};
\\
\implies &i_{L2}(\text{simulated, steady-state}): \minus 0.1616 \ \mathsf{A};
\\
&i_{L2}(\text{estimated; }\textsf{Simulink}): \minus 0.15948 \ \mathsf{A};
\\
\implies &\textsf{error: } 1.3 \ \mathsf{\%}.
\end{align*}
\begin{align*}
i_{L2}: \textsf{ref: } \minus 2 \ \mathsf{V} \implies &i_{L2}(\text{simulated}): \minus 2.621 \times 10^{-3} \sim \minus 37.15 \times 10^{-3} \ \mathsf{A};
\\
\implies &i_{L2}(\text{simulated, steady-state}): \minus 19.89 \times 10^{-3} \ \mathsf{A};
\\
&i_{L2}(\text{estimated; }\textsf{hardware}): \minus 19.9 \times 10^{-3} \ \mathsf{A};
\\
\implies &\textsf{error: } 0.07 \ \mathsf{\%}.
\end{align*}
\end{comment}
%%%%%%%%%%%%%%%%%%%%%%%%%%%%%%%%%%%%%%%%%%%%%%%%%%%%%%%%%%%%%%%%%%%%%%%%%%%%%%%%
In explaining the lack of accuracy in estimating $v_{C1}$ it is suggested that the small-ripple approximation is not valid for this state: for step changes in reference, $v_{C1}$ undergoes the greatest change in magnitude of any of the system state variables in the circuit. Thus the approximation that the perturbation from the steady-state value ($\hat{v}_{C1}$) is small relative to the steady-state value ($V_{C1}$), as calculated according to the steady-state operating point the model is linearised about (Equation~(\ref{eqn:modelY}); $V_{C1}$ is first element in steady-state state vector $\boldsymbol{X}$), is not valid.
\newpar
It is suggested that errors in the equations relating to the inductor $L_1$ used in describing the circuit (i.e. those of Section~\ref{sec:circuitanalysis} and Appendix~\ref{apx:circuit_analysis}) are the cause of the lack of accuracy in estimating $i_{L1}$. This was noticed when the project was at an advanced stage and as such was not corrected before implementing the controller in the final product.
%%%%%%%%%%%%%%%%%%%%%%%%%%%%%%%%%%%%%%%%%%%%%%%%%%%%%%%%%%%%%%%%%%%%%%%%%%%%%%%%
\begin{figure}[H]
    \begin{framed}
    \captionsetup[subfigure]{justification = centering}
    \centering
    \begin{subfigure}[b]{0.8\textwidth}
    \centering\includegraphics[height = 6cm]{figures/estimation/ref_y_pulsetrain.pdf}
    \caption{Reference train and output voltage response}
    \label{fig:estimatingconditions1}
    \end{subfigure}
    \\[11pt]
    \begin{subfigure}[b]{0.45\textwidth}
    \centering
    \includegraphics[height = 6cm]{figures/estimation/vC1_vC1.pdf}
    \caption{$v_{C1}$}
    \label{fig:estimatingfirst}
    \end{subfigure}
    \hfill
    \begin{subfigure}[b]{0.45\textwidth}
    \centering
    \includegraphics[height = 6cm]{figures/estimation/vC2_vC2.pdf}
    \caption{$v_{C2}$; plots coincide exactly at this scale}
    % \label{}
    \end{subfigure}
    \\[11pt]
    \begin{subfigure}[b]{0.45\textwidth}
    \centering
    \includegraphics[height = 6cm]{figures/estimation/iL1_iL1.pdf}
    \caption{$i_{L1}$}
    % \label{}
    \end{subfigure}
    \hfill
    \begin{subfigure}[b]{0.45\textwidth}
    \centering
    \includegraphics[height = 6cm]{figures/estimation/iL2_iL2.pdf}
    \caption{$i_{L2}$}
    \label{fig:estimatinglast}
    \end{subfigure}
    \end{framed}
    \vspace*{-8mm}
    \caption{Reference train of Figure~\ref{fig:estimatingconditions1} used in generating data of Figures~\ref{fig:estimatingfirst} through~\ref{fig:estimatinglast}}
    \label{fig:estimating}
\end{figure}
%%%%%%%%%%%%%%%%%%%%%%%%%%%%%%%%%%%%%%%%%%%%%%%%%%%%%%%%%%%%%%%%%%%%%%%%%%%%%%%%
\begin{figure}[H]
    \begin{framed}
    \captionsetup[subfigure]{justification = centering}
    \centering
    \begin{subfigure}[b]{0.8\textwidth}
    \centering\includegraphics[height = 6cm]{figures/estimation/ref_y_ramp.pdf}
    \caption{Reference voltage, input voltage, and output voltage}
    \label{fig:estimatingconditions2}
    \end{subfigure}
    \\[11pt]
    \begin{subfigure}[b]{0.45\textwidth}
    \centering
    \includegraphics[height = 6cm]{figures/estimation/vC1_vC1b.pdf}
    \caption{$v_{C1}$}
    \label{fig:estimating2first}
    \end{subfigure}
    \hfill
    \begin{subfigure}[b]{0.45\textwidth}
    \centering
    \includegraphics[height = 6cm]{figures/estimation/vC2_vC2b.pdf}
    \caption{$v_{C2}$; plots coincide exactly at this scale}
    % \label{}
    \end{subfigure}
    \\[11pt]
    \begin{subfigure}[b]{0.45\textwidth}
    \centering
    \includegraphics[height = 6cm]{figures/estimation/iL1_iL1b.pdf}
    \caption{$i_{L1}$}
    % \label{}
    \end{subfigure}
    \hfill
    \begin{subfigure}[b]{0.45\textwidth}
    \centering
    \includegraphics[height = 6cm]{figures/estimation/iL2_iL2b.pdf}
    \caption{$i_{L2}$}
    \label{fig:estimating2last}
    \end{subfigure}
    \end{framed}
    \vspace*{-8mm}
    \caption{Reference and input voltages of Figure~\ref{fig:estimatingconditions2} used in generating data of Figures~\ref{fig:estimating2first} through~\ref{fig:estimating2last}}
    \label{fig:estimating2}
\end{figure}
%%%%%%%%%%%%%%%%%%%%%%%%%%%%%%%%%%%%%%%%%%%%%%%%%%%%%%%%%%%%%%%%%%%%%%%%%%%%%%%%
The response to poor performance of the observer regime was to disable observer state feedback (achieved by setting the LQR weights for system states ($v_{C1}$, $v_{C2}$, $i_{L1}$, and $i_{L2}$) to 0; weight affecting $x_i$ is the only non-zero entry in weight matrix).
%%%%%%%%%%%%%%%%%%%%%%%%%%%%%%%%%%%%%%%%%%%%%%%%%%%%%%%%%%%%%%%%%%%%%%%%%%%%%%%%
%%%%%%%%%%%%%%%%%%%%%%%%%%%%%%%%%%%%%%%%%%%%%%%%%%%%%%%%%%%%%%%%%%%%%%%%%%%%%%%%
\subsection{Hardware}
\subsection{Microcontroller breakout}

\subsection{Controller and ultra-capacitor charging}

\subsection{\'Cuk converter and driving}

%%%%%%%%%%%%%%%%%%%%%%%%%%%%%%%%%%%%%%%%%%%%%%%%%%%%%%%%%%%%%%%%%%%%%%%%%%%%%%%%
%%%%%%%%%%%%%%%%%%%%%%%%%%%%%%%%%%%%%%%%%%%%%%%%%%%%%%%%%%%%%%%%%%%%%%%%%%%%%%%%
\subsection{Programming}
\input{08b_discussion_Programming.tex}
%%%%%%%%%%%%%%%%%%%%%%%%%%%%%%%%%%%%%%%%%%%%%%%%%%%%%%%%%%%%%%%%%%%%%%%%%%%%%%%%
%%%%%%%%%%%%%%%%%%%%%%%%%%%%%%%%%%%%%%%%%%%%%%%%%%%%%%%%%%%%%%%%%%%%%%%%%%%%%%%%
% \subsection{Results}
%%%%%%%%%%%%%%%%%%%%%%%%%%%%%%%%%%%%%%%%%%%%%%%%%%%%%%%%%%%%%%%%%%%%%%%%%%%%%%%%
%%%%%%%%%%%%%%%%%%%%%%%%%%%%%%%%%%%%%%%%%%%%%%%%%%%%%%%%%%%%%%%%%%%%%%%%%%%%%%%%
%%%%%%%%%%%%%%%%%%%%%%%%%%%%%%%%%%%%%%%%%%%%%%%%%%%%%%%%%%%%%%%%%%%%%%%%%%%%%%%%